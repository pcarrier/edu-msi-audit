\section{Outils existants}
\subsection{Vue d'ensemble}
\subsection{Sous-systèmes}
\subsubsection{Comptabilité}
\subsubsection{Ressources humaines}
\subsubsection{Adhérents et multidiffusion}
\subsubsection{Grand Café}
\subsubsection{Assistance informatique}

L'outil Assistance informatique est utilisé par l'ENE.

Il a plusieurs vocations :
\begin{itemize}
\item Suivre des assistances en cours, que ce soit par téléphone,
      courriel ou sur place, notamment en cas de changement d'opérateur ;
\item Remonter des statistiques au système de direction, par exemple les temps
      moyens de traitement, y compris par opérateur, le nombre de traitements,
      etc.
\item Remonter des statistiques aux partenaires publics sur les problèmes
      rencontrés par les usagers dans l'utilisation de leurs outils, notamment
      pour le Bureau Virtuel ;
\item Constituer une base de connaissances centralisée en standardisant les
      procédures d'assistance.
\end{itemize}

Il s'agit d'une application Web développée en interne quand le service était
assuré directement par la Direction des Systèmes d'Information de Grenoble,
Université de l'Innovation, et sur lequel EVE n'a aucun contrôle.

Son interface est présentée en annexe \ref{gestion_incidents}.

\subsubsection{LocaMIPE}
\subsubsection{Pépinière}
Assos, matos, salles, demandes, événements
\subsubsection{API ?}
\subsubsection{Documentation}
