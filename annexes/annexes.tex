\section{Cadre de l'organisation}
\section{Statuts}
\section{D�l�gation de Service Public}
\section{Diagrammes de gestion}
\subsection{Organigramme}
\subsection{Diagramme de flux}
\subsection{Exemples de commandes}
\section{Illustration des outils}
\section{Mod�les et exemples de pi�ces}
\subsection{A}
\subsection{Cha�ne de fin de stock au Grand Caf�}

\begin{center}
\begin{dot2tex}[circo,tikzedgelabels]
  digraph FinStockBar {
  node [ shape = "circle" ];
  SI -> S [ label = "1" ] ;
  S -> R [ label = "2" ];
  R -> F [ label = "3" ];
  F -> S [ label = "4" ];
  S -> SI [ label = "5" ];
  S -> C [ label = "6" ];
  C -> SI [ label = "7" ];
  C -> F [ label = "8" ];
  C -> SI [ label = "9" ];
}
\end{dot2tex}
\end{center}

\begin{itemize}
\item Le SI (caisse du Grand Caf�) indique au S (serveur) une fin de stock ;
\item Le S transmet l'information au responsable stocks (R) ;
\item Le R passe une commande au fournisseur (F) ;
\item F livre le produit, fournit un bon de livraison et une facture au S ;
\item Le S met � jour le stock dans le SI ;
\item Le S transmet le bon de livraison et la facture au comptable (C) ;
\item Le S saisit la facture dans le SI ;
\item Le S effectue le r�glement (8) aupr�s du F ;
\item Le S l'indique au SI.
\end{itemize}

