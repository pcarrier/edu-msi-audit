\section*{Introduction}

\textit{�~Comment r�pondre aux attentes de la D�l�gation de Service Public li�es � la P�pini�re d'association ?~�}

Cet audit, �tabli dans le cadre du cours de Management des Syt�mes d'Information de M. Spinard � l'Universit� Joseph Fourier, est issu d'une s�rie de rencontres avec l'�quipe d'�ponyme, pr�sent�e en \ref{presentation}. De services en services, mais �galement aupr�s des b�n�voles, nous avons souhait� d�terminer quels outils s'int�graient au fonctionnement du b�timent, que ce soit dans les missions r�gies des contrats et conventions, pr�sent�s en \ref{contrats}, ou organis�es selon les volont�s de l'association. Il s'est av�r� que non seulement ces outils �taient peu fonctionnels et efficaces, les informations fortement redondantes mais insuffisantes, mais que la structuration, pr�sent�e en \ref{structuration} et les r�gles de gestion, pr�sent�es en \ref{gestion}, �taient mal ou pas d�finies, et pas document�es.

Devant le besoin exprim� de formalisation, nous avons donc choisi d'utiliser ce document comme l'�bauche d'un prochain document de r�f�rence pour l'association, d�passant les attentes du cahier des charges de l'audit.