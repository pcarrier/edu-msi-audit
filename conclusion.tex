\section*{Conclusion}

Depuis 5 ans, Éponyme constitue une expérience unique.
Son mode de gestion se construit par tâtonnements, et elle
tarde à adopter les changements imposés par sa croissance.
Le spectre de ses activités s'élargit au gré des opportunités.
Devant ce contexte dynamique où aucune formalisation n'est pérenne,
qu'il s'agisse par exemple de structuration des équipes ou de standardisation des procédés,
il convient de concevoir des outils évolutifs et interopérables.
Devant la faible prévisibilité des activités,
notamment en matière de financement ou de succès,
il convient de multiplier les outils de suivi et d'alerte.
Nous l'avons défendu, Éponyme doit concevoir des outils de retour qualité.
Il est impératif de faire du SI l'un de leurs objets, dans la recherche
d'un compromis entre rigueur et souplesse, puisqu'\textit{a contrario},
le déploiement d'un SI contraignant, principalement au niveau opérationnel,
peut pousser l'organisation à mieux se définir,
l'aidant ainsi à gagner en lisibilité et en crédibilité.

\textit{«~La route est longue, mais la voie est libre.~»}